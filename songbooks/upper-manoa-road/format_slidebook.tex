\documentclass[letterpaper,oneside,landscape]{article}
\usepackage[pdfpagemode=UseThumbs]{hyperref}
\usepackage{color}
\usepackage[slides]{../_styles/songs}
% \includeonlysongs{2}

\usepackage{setspace}
\usepackage[utf8]{inputenc}
\usepackage{etoolbox}

% turn off chords for all slides
\providebool{gchords}
\setbool{gchords}{false}

%\doublespacing

\setlength{\oddsidemargin}{-0.5in}
\setlength{\evensidemargin}{-0.5in}

%\setlength{\topmargin}{-0.75in}
%\setlength{\topmargin}{0in}
%\setlength{\topskip}{1in}
%\setlength{\headheight}{13.6pt}
%\setlength{\headsep}{0.1in}

\setlength{\textheight}{6.5in}
\setlength{\textwidth}{10in}

%\centering

% Don't number the verses:
\noversenumbers

% Put each verse and chorus on a separate slide:
\sepverses

% Use 28pt Adobe Helvetica font for the lyrics
\renewcommand{\lyricfont}{%
  \fontfamily{phv}\fontseries{b}\fontsize{38pt}{38pt}\selectfont%
}

% If background colors are supported on this machine, then slides
% will have white lettering on a blue background.
\csname @ifundefined\endcsname{set@page@color}{}{
  \definecolor{SlideBG}{rgb}{0,0,0.13} %0.43
  \pagecolor{SlideBG}
  \color{white}
  \definecolor{SongbookShade}{rgb}{0,0,0.2}
}

% Define some headers for each slide to help the projector-operator find the correct slide.  We use the fancyhdr package for this.
\IfFileExists{fancyhdr.sty}{
  \usepackage{fancyhdr}
  \usepackage{extramarks}
  \pagestyle{fancy}
  \fancyhf{}
  \lhead{\sffamily\firstleftmark}
  \rhead{\sffamily\firstrightmark}
  \renewcommand{\headrulewidth}{0pt}

  \renewcommand{\songmark}{\markboth{}{\thesongnum}}
  \renewcommand{\versemark}{%
    \ifvnumbered
      \markboth{\thesongnum. \songtitle}{Verse \theversenum}%
    \else
      \markboth{\thesongnum. \songtitle}{}%
    \fi
  }
  \renewcommand{\chorusmark}{\markboth{\thesongnum. \songtitle}{Chorus}}
}{}


% load the generic song commands 
%----------------------------------------------------------
% define song info commands
% music anf lyrics by
\newcommand{\musicLyricsBy}{} 
\newsongkey{mlby}{\def\musicLyricsBy{}}
	{\def\musicLyricsBy{\sffamily\it\small letra e música por #1\par}}

% music by
\newcommand{\musicby}{} 
\newsongkey{music_by}{\def\musicby{}}
	{\def\musicby{\sffamily\it\small música por #1\par}}

% old one - delete?
%\newsongkey{mlby} {\def\musicLyricsBy{}}
%	{\def\musicLyricsBy{\sffamily\it\small letra e música por #1\par}}
	
% lyrics by
\newcommand{\lyricsby}{}
\newsongkey{lyrics_by}{\def\lyricsby{}}[\def\lyricsby{}]{\def\lyricsby{\sffamily\it\small letra por #1\par}}

% arranjement by
\newcommand{\arrangementby}{} 
\newsongkey{arrangement_by}{\def\arrangementby{}}
	{\def\arrangementby{\sffamily\it\small arranjo por #1\par}}

% bible verse
\newcommand{\bibleverse}{} 
\newsongkey{bible_verse}{\def\bibleverse{}}
	{\def\bibleverse{\sffamily\it\small Verso Biblico #1\par}}

% licence or copyright
\newcommand{\licenceorcopyright}{} 
\newsongkey{licence_or_copyright}{\def\licenceorcopyright{}}
                 {\def\licenceorcopyright{\sffamily\it\small Copyright  #1\par}}


% licence
\newcommand{\licence}{} 
\newsongkey{licence}{\def\licence{}}
	{\def\licence{\sffamily\it\small Licence #1\par}}        
                 
% video url
\newcommand{\videourl}{} 
\newsongkey{video_url}{\def\videourl{}}
                 {\def\videourl{\sffamily\it\small Video URL  #1\par}}


% title_original
\newcommand{\titleoriginal}{} 
\newsongkey{title_original}{\def\titleoriginal{}}
                 {\def\titleoriginal{\sffamily\it\small Título Original:  ``#1"\par}}



% number
\newcommand{\psalterionumber}{} 
\newsongkey{psalterio_number}{\def\psalterio_number{}}
                 {\def\psalterio_number{\sffamily\it\small Psaltério #1\par}}

% extra
\newcommand{\extra}{} 
\newsongkey{extra}{\def\extra{}}
                 {\def\extra{\sffamily\it\small #1\par}}

% id
\newcommand{\id}{} 
\newsongkey{id}{}{}

